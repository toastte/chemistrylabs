\documentclass{article}

\usepackage{mathtools}
\usepackage{amssymb}
\usepackage{graphicx}
\usepackage{titling}
\usepackage[table,xcdraw]{xcolor}
\usepackage[letterpaper, portrait, margin=1.5in]{geometry}
\usepackage{float}
\usepackage{units}
\usepackage{color}

\begin{document}

\begin{titlepage}

\title{\huge Relative Mass with Beans\vskip 0.5em}

\pretitle{\begin{center}\par}
\posttitle{\large \par 6th Period \vskip 0.5em\end{center}}

\author{Fletcher Collins}
\maketitle

\end{titlepage}

\section{Purpose}

	The purpose of this lab was to find the relative mass of four different types of beans and then to apply the relative mass to how atomic mass is measured.

\section{Methods/Materials}

	No changes were made to the original procedure or materials used in this lab. They may be found in the lab printout.

\section{Data}

	The data collected during this lab can be found on the printout.

\section{Conclusion}

	During this lab, we measured the masses of each bean in grams and then used those measurements to find relative mass. Atomic mass also uses relative mass, with one-twelfth of a carbon-12 atom being used as the unit. In our lab, we used navy beans, which had the smallest mass, as our unit. This scheme of mass allows for easy comparison of different elemetns, or in our case, beans. For instance, if an element is 2 units, we know it is equal to twice the base unit.

\section{Lab Questions}

	The lab questions can be found on the lab printout.

\end{document}