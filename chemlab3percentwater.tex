\documentclass{article}

\usepackage{mathtools}
\usepackage{amssymb}
\usepackage{graphicx}
\usepackage{titling}
\usepackage[table,xcdraw]{xcolor}
\usepackage[letterpaper, portrait, margin=1.5in]{geometry}
\usepackage{float}
\usepackage{units}
\usepackage{color}

\begin{document}

\begin{titlepage}

\title{\huge \% of Water in Copper(II) Sulfate Pentahydrate \vskip 0.5em}

\pretitle{\begin{center}\par}
\posttitle{\large \par 6th Period \vskip 0.5em\end{center}}

\author{Fletcher Collins}
\maketitle

\end{titlepage}

\section{Purpose}

Our reasons for doing this lab was to determine the \% of water in a hydrate. Copper(II) Sulfate Pentahydrate was used in this lab

\section{Methods/Materials}

Refer to the lab printout for the procdure and materials.

\section{Data}

The data for this lab can be found on the printout.

\section{Conclusion}

The percent error for the percentage of water in our sample was a notably high amount. While we were heating the material, it is likely it was removed from the heat before all of the water had been vaporized. A faint blue color was observed in the material even after it was removed from the heat. This small amount of water is likely the reason for our high percent error. In future labs, being sure that a reaction is completed before the mass is measured will keep this type of error from occuring again.

\section{Lab Questions}
\begin{enumerate}
\item Copper(II) sulfate pentahydrate is a fine powder with a light blue color.
\item The water vaporizing from the heat is likely the cause of this sound.
\item The compound has the same fine powder look as before, but the blue color has changed to a white.
\item The formation of the anhydrous compound is a chemical change, because the compound was heated to cause the change.
\item When the drops of water were added, a slight sizzling was heard, most likely because the material had been recently heated.
\item The two masses of the compound had a difference of around 1.17 grams. I would expect these two masses to be almost the same.
\item Our conclusions on this lab may be found in the conclusion section. 
\end{enumerate}
\end{document}
