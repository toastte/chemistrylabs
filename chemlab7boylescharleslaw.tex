\documentclass{article}

\usepackage{mathtools}
\usepackage{amssymb}
\usepackage{graphicx}
\usepackage{titling}
\usepackage[table,xcdraw]{xcolor}
\usepackage[letterpaper, portrait, margin=1.5in]{geometry}
\usepackage{float}
\usepackage{units}
\usepackage{color}

\begin{document}

\begin{titlepage}

\title{\huge Gas Laws \vskip 0.5em}

\pretitle{\begin{center}\par}
\posttitle{\large \par 6th Period \vskip 0.5em\end{center}}

\author{Fletcher Collins}
\maketitle

\end{titlepage}

\section{Purpose}

	This lab was done to discover the different relationships between volume vs pressure or temperature and express them through a mathematical relationship.

\section{Methods/Materials}

	A simulation was used to find the data during this lab. The procedure may be found on the handout.

\section{Data}

	The data for this lab may be found either in a table on the handout or plotted on different graphs.

\section{Conclusion}

	For the first section of this lab, we looked at the relationship between pressure and volume of a gas with Boyle's law. When pressure was graphed vs volume, it appeared that an inverse relationship existed between the two. This became even more clear when pressure was graphed against the inverse of volume, because the curve was cancelled out. This would make sense mathematically because  \(\frac{1}{1/x}=x\). Essentially, the most basic mathematical relationship between volume and pressure is \(V=\frac{1}{P}\), where \(V\) is volume and \(P\) is pressure.
\\\\
	The next section of this lab involved finding the relationship between volume and temperature. When temperature and volume were plotted against eachother, it became clear that the relationship between the two is directly proportional. It meets both the criteria for this relationship, one being that the line of best fit is linear, and that it crosses through the origin of the graph. Simply put, as volume increases, temperature increases.

\end{document}