\documentclass{article}

\usepackage{mathtools}
\usepackage{amssymb}
\usepackage{graphicx}
\usepackage{titling}
\usepackage[table,xcdraw]{xcolor}
\usepackage[letterpaper, portrait, margin=1.5in]{geometry}
\usepackage{float}
\usepackage{units}
\usepackage{color}

\begin{document}

\begin{titlepage}

\title{\huge Mole Ratios and Reaction Stoichiometry \vskip 0.5em}

\pretitle{\begin{center}\par}
\posttitle{\large \par 6th Period \vskip 0.5em\end{center}}

\author{Fletcher Collins}
\maketitle

\end{titlepage}

\section{Purpose}

	The purpose of this lab is to experimentally determine the mole-to-mole ratios between the reactants and products in two double displacement reactions.

\section{Methods/Materials}

	The methods and materials to this lab may be found on the printout.

\section{Data}

	The data found during the lab can be found on the printout.

\section{Conclusion}

	The experimental mole ratio found during this lab was extremely close to the theoretical value. So close, in fact, that both are virtually the same. The experimental yields found were very close when compared to the theoretical yields. To explain our very low error, we were very sure to follow every rule very closely when following the lab procedure. The crucibles were never touched by bare hands, they were kept on the Bunsen Burner as long as possible, and we took our measurements very carefully. Another possible explanation for our data's accuracy is just by chance.

\end{document}