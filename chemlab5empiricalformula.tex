\documentclass{article}

\usepackage{mathtools}
\usepackage{amssymb}
\usepackage{graphicx}
\usepackage{titling}
\usepackage[table,xcdraw]{xcolor}
\usepackage[letterpaper, portrait, margin=1.5in]{geometry}
\usepackage{float}
\usepackage{units}
\usepackage{color}

\begin{document}

\begin{titlepage}

\title{\huge Empirical Formula of Magnesium\vskip 0.5em}

\pretitle{\begin{center}\par}
\posttitle{\large \par 6th Period \vskip 0.5em\end{center}}

\author{Fletcher Collins}
\maketitle

\end{titlepage}

\section{Purpose}

	The purpose of this lab was to find the empirical formula of magnesium oxide and to compare the value found to the known value.

\section{Methods/Materials}

	No changes were made to the original procedure or materials used in this lab. They may be found in the lab printout.

\section{Data}

	The data collected during this lab can be found on the printout.

\section{Conclusion}

	The \% error found in our lab was too large to be connected to random error. The most likely cause of this was an error in our measurement when finding the mass of the magnesium oxide. Another possibily could be that not all of the oxidized magnesium was removed with sandpaper before it was heated. The fact that there was triple the oxygen atoms then there was supposed to be contributes to this possibility. In the future, we will need to be sure that instructions are followed as thoroughly as possible.

\section{Lab Questions}

	The lab questions can be found on the lab printout.

\end{document}