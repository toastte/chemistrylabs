\documentclass{article}

\usepackage{mathtools}
\usepackage{amssymb}
\usepackage{graphicx}
\usepackage{titling}
\usepackage[table,xcdraw]{xcolor}
\usepackage[letterpaper, portrait, margin=1.5in]{geometry}
\usepackage{float}
\usepackage{units}
\usepackage{color}

\begin{document}

\begin{titlepage}

\title{\huge Gas Laws Lab \vskip 0.5em}

\pretitle{\begin{center}\par}
\posttitle{\large \par 6th Period \vskip 0.5em\end{center}}

\author{Fletcher Collins}
\maketitle

\end{titlepage}

\section{Purpose}

	This lab was done to apply our knowledge of stoichiometry and the ideal gas law to calculate the amount of gas formed in a chemical reaction, as well as the percent error.

\section{Methods/Materials}

	The procedure and materials are unchanged from the lab handout.

\section{Data}

	The data found during this lab may be found on the printout.

\section{Conclusion}

	Using the ideal gas law and stoichiometry was helpful practice, and helped to understand how they're applied in a lab setting. In addition, this lab gave a basic look at how materials in a gaseous state need to be handled differently when compared to solids or liquids. It is likely that being new to handling gases increased our percent error. An observation to note was that the balloon filled with CO\textsubscript{2} felt heavier compared to a normal balloon, which was interesting in that it is uncommon to feel the weight of a gas.

\end{document}